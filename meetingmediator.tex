\section{Meeting Mediator}
Autor: Antje Westerhoff

\subsection{Idee und Motivation}
MeetingMediator(MM) wurde als System zur Unterstützung von Gruppendiskussionen
in gleichen und verteilten Arbeitsumgebungen entwickelt. Das System liefert ein
real-time feedback auf der Basis von Daten, die mittels soziometrischer
Aufzeichnungen (Sociometric badges) ermittelt werden.
Insbesondere bei Diskussionen mit verteilten Gruppenmitgliedern, fehlt der
Gruppe teilweise die Möglichkeit auf die verschiedenen soziologischen Signale
der anderen Gruppenmitglieder zu reagieren. Die Dominanz einzelner Mitglieder
führt zwar einerseits die Diskussion voran, andererseits werden weniger
dominante Diskussionsteilneher entmutigt, oder deren Beiträge unterdrückt.
Die Sociometric badges sammeln und analysieren verbale und non-verbale
Verhaltensmerkmale. Mittels MM werden diese Daten ausgewertet und den jeweiligen Gruppenmitgliedern auf deren Mobiltelefon visualisiert. Dieses Vorgehen soll dem einzelnen Diskussionsteilnehmer eine wenig ablenkende Information bieten, das eigene Verhalten im Kontext zum
Gruppenverhalten einzuordnen und daraus möglicherweise Verhaltensänderungen
 abzuleiten, um den Gruppenerfolg zu vergrößeren.

\subsection{Adaption}
MM verwendet dezentrale Einheiten zur Aufzeichnung und Auswertung soziologischer
Daten zur Verbessung von Gruppendiskussionen.
Die Länge und Häufigkeit von Gesprächsbeiträgen und die Bewegung und Hinwendung
zu anderen Gruppenmitgliedern charakterisiert die Dominanz von
Diskussionsteilnehmern.

Um Teilnehmern an einer Gruppendiskussion zusätzliche
Informationen über soziologische Signale zur Verfügung stellen zu können, werden Instrumente zur Messung dieser Signale benötigt. Die Entwickler von MM wählen dazu mobile
Sensoren (Sociometric badges), die jeder Teilnehmer um den Hals tragen kann.
Die Geräte erkennen Audio-Signale und messen mittels Infrarot-Sensoren
Bewegungen. Mittels Bluetooth werden die Messdaten an das Mobiltelefon des
Teilnehmers übermittelt. Die Sociometric badges kommunizieren untereinander über
2.4 GHz radio.

Die so übermittelten Daten werden dann durch die
MeetingMediator-Applikation akkumuliert und visualisiert.
Die graphische Anzeige auf den Mobiltelefonen der Teilnehmer wird in 5 Sekunden
Intervallen aktualisiert und kumuliert die Daten über den gesamten Zeitraum der Diskussion.
Die Anzeige auf den Mobiltelefonen der Diskussionsteilnehmer wurde gewählt, um
nicht einen weiteren möglicherweise ablenkenden Kommunikationskanal einzuführen.

Die Grafik stellt die einzelnen Mitglieder als Rechtecke in
unterschiedlichen Farben, korrespondierend zu den Farben der Sociometric badges,
dar. Die Balance der Diskussion wird als zentraler Kreis dargestellt, der wie
ein Ball an elastischen Seilen zu den Rechtecken der Diskussionsteilnehmer
gezogen wird. Die Dicke der Linie, die den zentralen Kreis mit den Rechtecken
verbindet, visualisiert die Sprechzeit des jeweiligen Teilnehmers. 
---Grafik Figure 2. einfügen

\subsection{Evaluation}
MM möchte die Effizienz von Gruppendiskussionen erhöhen, indem die
Gruppendynamik in Echtzeit graphisch dargestellt wird und damit die Änderung von 
 individuellen und Gruppenverhaltensweisen fördert.
 
Die Studie zeigte, dass:
MM dazu beiträgt, dass weniger überlappende Gespräche geführt werden, d.h. die
Gruppe teilt sich nicht in Untergruppen, sondern diskutiert gemeinsam.

MM zu kürzeren Sprechsequenzen führt. Es findet insbesondere in verteilten
Umgebungen eine höhere Interaktion statt.

die Nutzung des allgegenwärtigen Mobiltelefons als Anzeigemedium die
Studienteilnehmer weniger ablenkt und stört als eine für alle sichtbare
zusätzliche Anzeige.

Die von den MM Entwicklern aufgestellten Hypothesen zur Identifikation und
Beeinflussung von dominanten und zurückhaltenderen Gruppenmitgliedern zur
Förderung des kooperativen Arbeitens, sind nicht so eindeutig bestätigt worden,
wie die zuvor beschrieben Effekte.

Durch die Messung der Sprechzeit identifizerte MM dominante Gruppenmitglieder
genauso, wie es auch intuitiv erfolgen würde. Es fand sich aber kein
signifikanter Unterschied zwischen dominanten und weniger dominaten
Gruppenmitgliedern bei der Messung von Bewegungen.
Der Einfluss, den MM auf das Verhalten von dominaten und weniger dominanten
Diskussionsmitgliedern hat ist sehr unterschiedlich.
Bei der Beobachtung ohne die Beeinflussung von MM zeigte sich, dass weniger
dominante Personen in Gruppen mit einer dominanten Person insbesondere in
verteilten Arbeitsgruppen häufiger das Wort ergreifen, als in Gruppen mit
ausschließlich wenig-dominaten Personen.

Die Hypothese, dass MM die Gruppendiskussionen in verteilten Umgebungen
wesentlich mehr beeinflusst, als in gemeinsamen Umgebungen, konnte nicht
allgemein bestätigt werden. Hier wurde eher ein Zusammenhang mit dem
Vorhandensein von dominaten Personen in der Gruppendiskussion hergestellt.
Insbesondere in verteilten Umgebungen fällt es weniger-dominaten Personen mit
MM leichter sich in die Diskussion einzubringen, da MM den dominanten
Gruppenmitgliedern die fehlende Balance in der Diskussion optisch vermitteln
kann. 
Der Einsatz von MM zeigte, dass man durch die Visualierung des Gruppenverhaltens
den Unterschiede im Verhalten von dominanten und weniger dominanten
Diskussionsteilnehmern verringern kann.

\subsection{Ergebnis}
MM ist ein mobiles System, das ein real-time feedback in Gruppendiskussionen
liefert, mit dem Ergebnis, dass die Zusammenarbeit verbessert wird.
Dieser Effekt macht sich insbesondere in verteilten Gruppendiskussionen mit
dominanten Gruppenmitgliedern bemerkbar. Verteilte Diskussionen werden damit
lokalen Diskussionen angenähert.


