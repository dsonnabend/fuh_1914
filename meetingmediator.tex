\section{Meeting Mediator}
%Autor: Antje Westerhoff

\subsection{Idee und Motivation}
Meeting Mediator(MM) \cite{KimChaHolPent2008} wurde als System zur
Un\-ter\-stütz\-ung von Gruppendiskussionen in räumlich gleichen und verteilten
Arbeitsumgebungen entwickelt. Das System liefert ein real-time feedback auf der
Basis von Daten, die mittels soziometrischer Aufzeichnungen ermittelt werden.
Dieses Feedback soll Gruppen unterstützen, Diskussionen effektiver zu führen,
indem alle Mitglieder gleichermaßen an der Diskussion teilnehmen.

Diskussionsgruppen setzen sich meist aus unterschiedlich dominanten Mitgliedern
zusammen. Die Dominanz einzelner Gruppenmitglieder führt zwar einerseits die
Diskussion voran, andererseits werden weniger dominante Diskussionsteilneher 
entmutigt, oder deren Beiträge un\-ter\-drückt.

Insbesondere bei Diskussionen mit verteilten Gruppenmitgliedern, fehlt der
Gruppe teilweise die Möglichkeit auf die zusätzlich zum gesprochenen Wort
vorhandenen verschiedenen soziologischen Signale z.B. Enthusiasmus oder
Körperhaltung, der anderen Gruppenmitglieder zu reagieren.
Die von MM verwendeten soziometrischen Sensoren (Sociometric badges)
\cite{MITbadge} sammeln und analysieren verbale und non-verbale
Verhaltensmerkmale.

Mittels MM werden diese Daten ausgewertet und den jeweiligen Gruppenmitgliedern
auf deren Mobiltelefon visualisiert. Dieses Vorgehen soll dem einzelnen
Diskussionsteilnehmer eine wenig ablenkende Information bieten, das eigene
Verhalten im Kontext zum Gruppenverhalten einzuordnen und daraus möglicherweise
Ver\-hal\-tens\-än\-de\-run\-gen abzuleiten, um den Gruppenerfolg zu
vergrößeren.

\subsection{Adaption}

MM ist eine J2ME-Applikation für Smartphones. J2ME \cite{J2ME} ist die Micro
Edition von Java (Java ME), die speziell die Erstellung von Java-Anwendungen
ermöglicht, die auf kleinen Geräten mit begrenzter Speicher-, Anzeige- und
Leistungskapazität ausgeführt werden. Die zur Software-Ent\-wick\-lung zur
Verfügung stehende API (Application Programming Interface) und ein umfangreiches
SDK (Software Development Kit) ermöglichen eine komfortable und durch die
Nutzung der Java VM auch eine hardware-un\-ab\-hän\-gi\-ge Programmentwicklung.

Zu\-sätz\-lich werden Sociometric Badges als dezentrale Einheiten zur
Aufzeichnung und Auswertung soziologischer Daten verwendet. In der derzeitigen
Version werden nur die Sprechdauer und -häufigkeit und die Körperbewegung
ausgewertet.
Die Länge und Häufigkeit von Ge\-sprächs\-bei\-trä\-gen und die Bewegung und
Hinwendung zu anderen Gruppenmitgliedern charakterisieren die Dominanz von
Diskussionsteilnehmern.

Um Teilnehmern an einer Gruppendiskussion zusätzliche Informationen über
soziologische Signale zur Verfügung stellen zu können, werden Instrumente zur
Messung dieser Signale benötigt. Die Entwickler von MM wählen dazu mobile
Sensoren (Sociometric badges) (siehe Figure \ref{badge}), die jeder Teilnehmer
um den Hals tragen kann.
Die Geräte erkennen Audio-Signale und messen mittels Infrarot-Sen\-so\-ren
Bewegungen. Mittels Bluetooth werden die Messdaten an das Mobiltelefon des
Teilnehmers übermittelt. Aufgrund der Reichweite von Bluetooth wurde die
räumlich verteilte Gruppe nur durch einen Sichtschutz zwischen den
Gruppenteilnehmern simuliert.

\begin{figure}[htp]
\centering
\includegraphics[width=8cm]{sociometricbadge.jpg}
\caption{Soziometrischer Sensor (Sociometric Badge) \protect\cite{MITbadge}}
\label{badge}
\end{figure}

Die so übermittelten Daten werden dann durch die Meeting-Mediator-Applikation
akkumuliert und visualisiert.
Die graphische Anzeige auf den Mobiltelefonen der Teilnehmer wird in 5 Sekunden
Intervallen aktualisiert und kumuliert die Daten über den gesamten Zeitraum der
Diskussion.
Die Anzeige auf den Mobiltelefonen der Diskussionsteilnehmer wurde gewählt, um
nicht einen weiteren mög\-li\-cher\-wei\-se ablenkenden Kommunikationskanal
einzuführen.

Die auf dem Mobiltelefon angezeigte Grafik (siehe Figure \ref{display}) stellt
die einzelnen Mitglieder als Rechtecke in unterschiedlichen Farben,
korrespondierend zu den Farben ihrer Sociometric badges, dar. Die Balance der
Diskussion wird als zentraler Kreis dargestellt, der wie ein Ball an elastischen
Seilen zu den Rechtecken der Diskussionsteilnehmer gezogen wird, abhängig von
deren jeweiligem Anteil an der Diskussion. Die Dicke der Linie, die den
zentralen Kreis mit den Rechtecken verbindet, visualisiert die Sprechzeit des
jeweiligen Teilnehmers. Die Farbe des zentralen Kreises zeigt mit grün eine hohe
Interaktivität der Diskussion an und möchte mit weiß zu mehr Interaktion
ermuntern.

\begin{figure}[htp] 
\centering 
\includegraphics[width=8cm]{mm.jpg}
\caption{Benutzeroberfläche von Meeting Mediator \protect\cite{MITbadge}}
\label{display}
\end{figure}


\subsection{Evaluation}
MM möchte die Effizienz von Gruppendiskussionen er\-höh\-en, indem die
Gruppendynamik in Echtzeit graphisch dargestellt wird und damit die Änderung von
 individuellen und Gruppenverhaltensweisen fördert.
 
Die MM Entwickler Kim et. al. führten eine Studie \cite{KimChaHolPent2008} an 36
Gruppen mit je 4 Gruppenmitgliedern zur Bewertung von MM durch. Wegen eines
technischen Defekts eines Sociometric badges bei der Audio-Analyse, wurden
jedoch nur jeweils 3 Gruppenmitglieder zur tatsächlichen Auswertung
herangezogen.
Eine Hälfte, also 18 Teams, führten die Aufgaben mit Unterstützung durch ein
MM-Feedback durch, während die anderen 18 Gruppen als Kontrollgruppen ohne MM
arbeiteten.

Die Aufgaben aller Teams mussten jeweils in 2 unterschiedlichen Situationen
bearbeitet werden.
Die erste Situation bildete den Fall ab, dass alle Mitglieder des Teams sich zur
gleichen Zeit am selben Ort befinden. Die zweite Situation simulierte den Fall,
dass sich je 2 Mitglieder zur gleichen Zeit an unterschiedlichen Orten befinden.
Die räumliche Trennung wurde für die Studie nur durch eine Sichttrennung mittels
eines Vorhangs simuliert.
Unabhängig von der jeweiligen Situation sollten 2 unterschiedliche
Aufgabenblöcke bearbeitet werden. Im ersten Block ging es um die Ideensammlung
und im zweiten Block um die Pro\-blem\-lö\-sung von jeweils gleichen Problemen.

Zu jeder Situation wurde im Anschluss ein persönlicher Fragebogen mit einer
Fünf- Punkte Likert-Skala \cite{WP_Likert} zu den Fragebereichen eigener
Charakter, Gruppendynamik und Leistung der Gruppenmitglieder und, falls
verfügbar, zum Nutzen von MM, von jedem Gruppenmitglied bearbeitet.
 
Die Studie \cite{KimChaHolPent2008} zeigte, dass:

\begin{itemize}
  \item MM dazu beiträgt, dass weniger überlappende Gespräche geführt werden, d.h. die
Gruppe teilt sich nicht in Untergruppen, sondern diskutiert gemeinsam.
  \item MM zu kürzeren Sprechsequenzen führt. Es findet insbesondere in verteilten
Umgebungen eine höhere Interaktion statt.
  \item die Nutzung des allgegenwärtigen Mobiltelefons als Anzeigemedium die
Studienteilnehmer weniger ablenkt und stört als eine für alle sichtbare
zusätzliche Anzeige.

\end{itemize}

Die von den MM Entwicklern aufgestellten Hypothesen zur Identifikation und
Beeinflussung von dominanten und zurückhaltenderen Gruppenmitgliedern zur
Förderung des kooperativen Arbeitens, sind nicht so eindeutig bestätigt worden,
wie die zuvor beschriebenen Effekte.

Durch die Messung der Sprechzeit identifizierte MM dominante Gruppenmitglieder
genauso, wie es auch intuitiv erfolgen würde. Es fand sich aber kein
signifikanter Unterschied zwischen dominanten und weniger dominanten
Gruppenmitgliedern bei der Messung von Bewegungen.
Der Einfluss, den MM auf das Verhalten von dominanten und weniger dominanten
Diskussionsmitgliedern hat, ist sehr unterschiedlich.
Bei der Beobachtung ohne die Beeinflussung von MM zeigte sich, dass weniger
dominante Personen in Gruppen mit einer dominanten Person insbesondere in
verteilten Arbeitsgruppen häufiger das Wort ergreifen, als in Gruppen mit
ausschließlich wenig-do\-min\-ant\-en Personen.

Die Hypothese, dass MM die Gruppendiskussionen in verteilten Umgebungen
wesentlich mehr beeinflusst, als in räumlich nicht getrennten Umgebungen, konnte
nicht allgemein bestätigt werden. Hier wurde eher ein Zusammenhang mit dem
Vorhandensein von dominanten Personen in der Gruppendiskussion hergestellt.
Insbesondere in verteilten Umgebungen fällt es weniger-dominanten Personen mit
MM leichter sich in die Diskussion einzubringen, da MM den dominanten
Gruppenmitgliedern die fehlende Balance in der Diskussion optisch vermitteln
kann. 
Der Einsatz von MM zeigte, dass man durch die Visualierung des Gruppenverhaltens
die Unterschiede im Verhalten von dominanten und weniger dominanten
Diskussionsteilnehmern verringern kann.

\subsection{Ergebnis}
MM ist ein mobiles System, das ein real-time feedback in Gruppendiskussionen
liefert, mit dem Ergebnis, dass die Zusammenarbeit verbessert wird.
Dieser Effekt macht sich insbesondere in verteilten Gruppendiskussionen mit
dominanten Gruppenmitgliedern bemerkbar. Verteilte Diskussionen werden damit
lokalen Diskussionen angenähert.


