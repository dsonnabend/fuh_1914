% !TEX encoding = UTF-8 Unicode

\documentclass{seminarvorlage}

\usepackage[utf8]{inputenc}

\begin{document}

% Unbedingt angeben: Titel, Autoren, E-Mail
% Freiwillig: Adresse
\title{Seminar 1915 - Unterstützung von Gruppendiskussionen in Seminaren}
\numberofauthors{2}
\author{
  \alignauthor Daniel Sonnabend\\
    \email{hartmann@gummi.de}
  \alignauthor aw\\
    \email{xxx@xxx.de}
}

\maketitle

\abstract{Diese Abhandlung erläutert die Geschichte der Gummi\-bärchen\ldots}

\keywords{Süßigkeiten, Gelatine, Bär, Lebensmittelfarbe.}

\section{Einleitung}

\section{Und Erwachsene ebenso?}

\section{Zusammenfassung}


% Bibliographie entweder direkt hier eingeben...
\begin{thebibliography}{9}
\bibitem{acmcategories}
How to classify works using ACM's computing classification system.
\newblock \url{http://www.acm.org/class/how_to_use.html}.

\bibitem{Ivory2001}
M.~Y. Ivory and M.~A. Hearst.
\newblock The state of the art in automating usability evaluation of user
  interfaces.
\newblock {\em ACM Comput. Surv.}, 33(4):470--516, 2001.

\end{thebibliography}


\end{document}
