% !TEX encoding = UTF-8 Unicode

\documentclass{seminarvorlage}

\usepackage[ngerman]{babel}
\usepackage[utf8]{inputenc}
\usepackage{graphicx}

% Silbentrennung im Fehlerfall hier angeben.
\hyphenation{Grup-pen-dis-kus-sion}
\hyphenation{kol-lo-bar-i-tive}
\hyphenation{Klas-sen-situ-at-ion}
\hyphenation{Wahr-neh-mungs-eb-ene}
\hyphenation{Man-ag-ment-tech-nik-en}
\hyphenation{Meeting-Mediator}

\begin{document}

% Unbedingt angeben: Titel, Autoren, E-Mail
% Freiwillig: Adresse
\title{Unterstützung von Gruppendiskussionen in Seminaren}
\numberofauthors{2}
\author{
  \alignauthor D. Sonnabend\\
    \email{yyy@yyy.de}
    \and
  \alignauthor A. Westerhoff\\
    \email{xxx@xxx.de}
}

\maketitle

\abstract{ Wir vergleichen anhand der Systeme Tin Can und Meeting Mediator zwei
Ansätze zur Unterstützung und Verbesserung der Diskussion in Gruppen.
Die Gruppengröße divergiert zwischen vier und elf Personen und bildet damit
typische Seminarsituationen ab. Nach einer kurzen Begriffsbestimmung gehen wir
näher auf die Anforderungen ein und beschreiben, warum eine Gruppendiskussion
manchmal nicht so effektiv ist, wie sie es sein könnte und wie dieses durch eine
Steigerung der Partizipation aller Diskussionsteilnehmer verbessert werden kann.
Beide Systeme beschreiben wir detailliert in Aufbau, Umfang und Nutzung um
abschliessend einen direkten Vergleich herstellen zu können.}


\keywords{Computer vermittelte Kommunikation, Unterstützung von
Gruppendiskussion, Backchannel, Tin Can, Meeting Mediator}

\section{Einleitung}
Die Diskussion einer Gruppe, deren Mitglieder sich am selben Ort zur selben Zeit
gegenübersitzen, wird oft als optimale Kommunikationstechnik betrachtet. Harry
et al. \cite{HarGorSch2012} sprechen sogar von einem `Goldstandard`. Viele
Com\-puter-\-ge\-stützte Systeme versuchen, genau diesen
Kommunikationsstandard virtuell abzubilden und eine Dimension zu ersetzen z.B. den unterschiedlichen
Ort oder die fehlende audio-visuelle Kommunikation. Dabei wird oft unterschätzt,
dass dieser 'Goldstandard' ebenfalls weiter verbessert werden kann, denn auch
solche Diskussionen sind häufig nicht so effektiv, wie sie es sein könnten.
Dabei denkt sicherlich jeder an selbst erlebte Besprechungen, die oft auf Grund schlechter
Strukturierung bzw. durch Unterbrechungen wesentlich fragmentiert wurden oder
von einigen Teilnehmern dominiert wurden, so dass wichtige Aspekte in der
Diskussion nicht berücksichtigt wurden.

Einfache Managementtechniken wie Brainstorming, eine Tagesordnung und eine
Rednerliste können da sicherlich bereits helfen.
Die beiden vorgestellten Systeme Tin Can und Meeting Mediator gehen allerdings
noch einen Schritt weiter und versuchen synchron zur Diskussion bzw. dem Vortrag
eine weitere Wahrnehmungsebene zu schaffen.

% Im Falle des Systemes Tin Can
% handelt es sich um Textnotizen, welche das Brainstorming bzw. die Diskussion
% gestalten und führen sollen. Bei dem Meeting Mediator geht es eher um zeitnahe
% Visualisierung der Beteiligung jedes einzelnen Gruppenteilnehmers während der
% Gruppendiskussion. Beide System wollen die Beteiligung steigern und somit
% schneller und effektiver zu Lösungen kommen.

\section{Begriffsbestimmung}
\subsection{Gruppendiskussion}
Unter dem Begriff der Gruppendiskussion verstehen wir in diesem Zusammenhang
eine themenzentrierte Ge\-sprächs\-situa\-tion von mehr als 2 Personen mit dem
Ziel des Brainstorming oder der Problemlösung. Bei einer Gruppendiskussion befinden
sich die Teilnehmer zeitgleich an dem selben Ort oder räumlich verteilt
(virtuelle Gruppe).

\subsection{Awareness}
Awareness ist die Wahrnehmung eines anderen Individuums bzw. dessen Präsenz.
Dazu gehören nicht nur die Sinneswahrnehmungen wie sehen, hören und fühlen,
sondern auch insbesondere die Interpretation dieser Wahrnehmungen im Kontext zu
den eigenen Aktivitäten.
Mit der Herstellung von Awareness in kooperativen Umgebungen mö\-chte man alle
soziologischen Signale transportieren wie z.B. Emotionen und allen
Gruppenmitgliedern den gleichen Kenntnisstand vermitteln.

\subsection{Aufmerksamkeit}
Laut Durkheim \cite{Dur1974} sind wir in jeder Situation immer mehr oder weniger
durch externe Stimuli abgelenkt. Diese Stör\-un\-gen unterbrechen den fiktiven
Begriff der Aufmerksamkeit. Es bedarf also einer Anstrengung anderen Personen
seine Aufmerksamkeit mitzuteilen. Das kann durch verschiedenste soziale Aktionen
präsentiert werden wie z.B. Zuhören oder aber das Schreiben von Notizen.

\subsection{Dominanz}
Tritt ein Individuum aus einer Gruppe durch üb\-er\-durch\-schnitt\-liche
Beteiligung heraus und wirkt damit beherrschend im Diskussionsverlauf, sprechen
wir von Dominanz.
In den beiden vorliegenden Studien wird Dominanz (im Gegensatz zu der
allgemeinen Bedeutung) nicht im negativen Kontext gesehen. Viel mehr beschreibt
eine hohe Dominanz die Beteiligung in der Gruppe.
Die Steigerung der Dominanz bei zurückhaltenderen Gruppenmitgliedern wird oft
als Ziel von Studien genannt.

\subsection{Kanal (Channel)}
Ein Kanal beschreibt in den vorgestellten Studien mehrere Dinge.
\cite{BergKara2009-1}.
Erstens ist ein Kommunikationskanal der Weg einer Botschaft vom Sender zum
Empfänger. Dies ist in der Soziologie die Sprache oder non-verbal die
Kör\-per\-spra\-che.
Im technischen Sinn beschreibt ein Kanal den Über\-mitt\-lungs\-weg von
Nachrichten.
So ist z.B. die Sprache ein Kommunikationskanal, ein Chat ein textueller Kanal
zum Austausch von Informationen. Ein Kanal wird oft von einem einzelnen
Teilnehmer genutzt um viele Teilnehmer zu erreichen (Broadcast).
In Gruppendiskussionen gibt es häufig Front- und Backchannel. Über den
Frontchannel kommuniziert der Vortragende und die direkt an der Diskussion
beteiligten Gruppenmitglieder. Über den Backchannel werden jedoch häufig
zusätzliche Informationen ausgetauscht, wie z.B. Nebendiskussionen oder Notizen.
Häufig stören die Aktivitäten die über den Backchannel laufen, die eigenliche
Frontchannel-Diskussion.

\subsection{Bühne (Stage)}
Anders als Kanäle beschreiben Bühnen (engl. Stage) Orte, an denen
gleichberechtigt eine Performance stattfinden kann \cite{Goff1959}. Dabei ist
weniger der Durchsatz das entscheidende Kriterium. Vielmehr soll ein
zusätzlicher Raum geschaffen werden um Informationen 'abzulegen'. Als Analogie dient den
Autoren Harry et al. ein Restaurant mit Speisesaal (front performance) und Küche
(back performance).
An beiden Orten werden verschiedene Tätigkeiten von den gleichen handelnden
Personen 'performt' die zu einem gemeinsamen Ergebnis führen.

\subsection{Soziale Spiegel (Social Mirrors)}
Soziale Spiegel \cite{BergKara2009-2} sind Visualisierungen von sozialen
Gruppendynamiken in Echtzeit, die allen Gruppenmitgliedern zur Verfügung stehen,
mit dem Ziel Dynamiken zu ver\-änd\-ern. Visualisiert wird das eigene Verhalten
sowie das Verhalten aller anderen Gruppenmitglieder aus der Sicht einer dritten
Person. Dabei wird die Relation des Individuums zur Gruppe verdeutlicht. Dies
soll jedem Individuum Anreize bieten, sein Verhalten an den Gruppendurchschnitt
anzupassen. Verarbeitet werden insbesondere auditive und visuelle, aber
auch emotionale Signale.

\section{Problemanalyse \& Anforderungen}
In Arbeitsgruppen wie zum Beispiel Seminaren, münd\-lich\-en Diskussionen oder
Projektbesprechungen gibt es Barrieren, die die Partizipation aller
Mitglieder verhindern und damit die Effektivität der Gruppe einschränken.

Dadurch kommt die Gruppe entweder gar nicht bzw. nicht effektiv zu einem
best\-mög\-lich\-en Ergebnis. Durch Dominanzen in der Gruppe kommt es nicht zu
einer gleichberechtigten Beteiligung allr Gruppenmitglieder an der Diskussion.
Dies mindert die Themenvielfalt und den Ideenaustausch. Introvertierte
Per\-sön\-lich\-keit\-en werden nicht maximal gefördert.

Gefordert wird, Dominanzen zu erkennen, introvertierte Per\-sön\-lich\-keit\-en
zu couragieren und die Themen-/\-Ideen\-vielfalt zu maximieren.

\section{Tin Can}

\subsection{Idee und Motivation}
Das Tin Can System \cite{HarGorSch2012} wurde 2012 am MIT Media Lab entwickelt.
Dabei handelt es sich um ein Tablet basiertes kollobaritive System, welches in
einer typischen Klassensituation es ermögicht, Ideen und wei\-ter\-führ\-en\-den
Diskussionen zu sammeln.

Motiviert durch zahlreiche Studien, welche belegen, dass eher schüchterne
Persönlichkeiten von zusätzlichen kollobrorativen Werkzeugen profitieren, möchte
Tin Can die Teilnahme am Unterricht bzw. Vortrag fördern.

Die Ideensammlung bzw. Themensammlung für weitere Diskussionen wird sowohl live
genutzt, damit der Vortragende mehr auf das Auditorium eingehen kann und es so
zu einer verbesserten Moderation kommt, als auch um nach\-träglich eine Analyse
des Vortrages zu erhalten. So erhält zum Beispiel jeder Teilnehmer eine E-Mail
nach Ende des Vortrages mit einer Ideen- und Themenliste.

Dadurch dass der Dozent den Backchannel moderierend beobachtet, kann er auf
Themen dediziert eingehen oder 'schüchterne' Teilnehmer ermutigen Ihre privaten
Notizen vorzurtragen oder zu veröffentlichen.

Im Unterschied zu anderen Arbeiten (Referenzen!), ist bei TinCan der Backchannel
dazu gedacht, den Frontchanel, also den Vortragenden oder Moderierenden zu
beeinflussen und so der Diskussion eine neue Richtung zu geben.

Die meisten Arbeiten versuchen den Übergang zwischen Front- und Backchannel zu
optimieren, die Motivation bei Tin Can hingegen liegt hingegeen auf der
Einführung eines weiteren Komzeptes, die s.g. Stages.

Dabei sind Stages am ehesten als eine Art Bühne zu verstehen. Neben der
Hauptbühne (dem Dozenten) gibt es eine Nebenbühne, das Tin Can System.

Tin Can ist kein Broadcast System in dem ein Teilnehmer leicht viele Teilnehmer
erreichen kann (und evenentuell mit der dadurch entstehenden Dominanz andere
Teilnehmer unterdrückt), sondern eher eine Nebenraum, in dem viele
Persönlichkeiten mit Ihrer 'Zielgruppe' kommunizieren können. Das kann der
Frontchannel sein, aber auch eine private Idee oder aber ein Themenvorschlag.

Im Gegensatz zu der Channelmethapher sollen die Teilnehmer motiviert werden
viele kleine Beiträge beizusteuern und nicht große Broadcast.

Die Tin Can Studie geht auf Grundlage moderner Sozialpsychologie davon aus, dass
introvertierte Persönlichkeiten eher schriftlich an einer Diksussion teilnehmen,
da sie so vorher ihren Beitrag prüfen, korrigieren oder ganz verwerfen können.

Daraus schliessen sie, dass der goldene Standard der normalen Diksussion in
kleinen Seminargruppen verbessert werden kann, in dem man durch einen
schriftliche Beteiligungsmöglichkeit auch die introvertieren Gruppenmitglieder
zu Wort kommen lässt. Somit verbessert sich das Outcome der Diskussion.
 
\subsection{Architektur \& Technische Umsetzung}
Tin Can wurde als eine App für Tablets umgesetzt, da Tablets unauffällige
Begleiter im Alltag sind und weniger störend wirken wie ein aufgeklapptes
Laptop.

Durch die geringe Formgröße kann jedes Gruppenmitglied selbst
entscheiden wie es das Gerät halten möchte. Damit reduziert sich die
Wahrnehmung des Gerätes als eigenständigen Gegenstand. Es ist mehr wie ein
Schreibblock.

Tin Can wurde als Client-Server-Anwendung umgesetzt, die Clients sind reine
Tablets, der Server führt ein Archive und bietet nachgelagerte Dienste an, wie
z.B. den Mailversand mit einer Zusammenfassung der Diskussion nachdem diese
beendet wurde.

Bei Tin Can gibt es kein Rechtesystem, alle Benutzer sind gleichberechtigt.

Das System zeigt den aktuellen Stand der Diskussion. Der Bildschirm zeigt 3
Hauptbereiche: eine Themenliste, eine analoge Uhrzeit und eine Ideenliste.

Die Themenliste sammelt sowohl alte, aktuellen und vorgeschlagene Themen, jeder
Benutzer kann einfach mit einem Button 'Theme hinzufügen' ein neues Thema
erstellen. Jedes Thema wird mit einem Kurztext beschrieben und mit einer Farbe
von anderen Themen abgegrenzt. Anhand einer kleinen Uhr kann man die Dauer der
diskutierten Themen sehen, bzw. die aktuelle Dauer bei dem aktuellen Thema.

Indem man ein Thema antippt, kann man in einem Untermenü den Status des Themas
verändern.

Die analoge Uhr zeigt erstmal die aktuelle Uhrzeit, sowie farblich markiert die
Dauer der einzelnen Themen.
Wenn eine Stunde verstrichen ist, werden die Segemente welche die Dauer
visualisieren aus dem Zentrum der Uhr an ihre Ecken ausgelagert und die aktuelle
Stunde wird im Zentrum dargestellt. So erreicht Tin Can eine maximale
Sitzungsdauer von 4 Std. ohne dass die Visualisierung unübersichtlich wird.

Die Liste der Ideen befindet sich auf der rechten Seite des Bildschirmes, die
Ideen sind chronologisch sortiert. Ideen sind einfach Textbeiträge wie Fragen,
Aussagen oder Twitter-ähnliche Antworten. Man kann mit zwei Buttons eine Idee
hinzufügen bzw. zur Gruppe hinzufügen. Erster Button speichert die Idee in einem
persönlichen Bereich, die anderen pupliziert sofort die Idee im Zeitraster der
Gruppenideen und zu dem persönlichen Bereich. Ideen welche im Gruppenraster
dargestellt werden, weisen in Klammern den Autor des Beitrages aus. Außerdem
gibt es ein Bewertungssystem, mit welchem andere Teilnehmer Ihre Zustimmung
signalisieren können. Wie man auch in anderen Studien gesehen hat, so nutzt auch
Tin Can nur ein positives Feedback in Facebook-Form (Likes).

Benutzer werden am Rand des Bildschirm nach einem Zufallsprinzip als Rechtecke
mit Ihrem Namen dargestellt. Mit einem Tipp auf diese Rechteck kann man alle
Ideen (auch die persönlichen) des Benutzer sehen. Jeder (nicht nur der
Ersteller) kann eine Idee eines anderen Benutzer aus dessen persönlichen
Bereiches in die Gruppenliste der Ideen ziehen. Diese Idee wird dann
entsprechend gekennzeichnet ('Eine Idee von Daniel veröffentlich von Antje).

Damit sind persönliche Ideen per Definition nichts privates. Sie bieten dem
Benutzer eher eine Möglichkeit, seine Zurückhaltung bei einer Idee auszudrücken.

\subsection{Adaption}
Tin Can führt dazu so genannte `Stages` ein. Diese sind am ehesten mit dem Bild
einer Theaterbühne zu vergleichen. Der Sprecher steht dabei vor der Gruppe auf
der so genannten Haupt-Bühne(`main spoken stage). Die Zuhörer haben in der
normalen Vortragssituation keine Möglichkeit im Vortag Präsenz zu zeigen ohne
unterbrechend einzugreifen. Dazu bietet TinCan eine Nebenbühne bzw. einen
zweiten Kanal (`backchannel`) in welchem die Zuhörer Ihre Ideen (privat wie
öffentlich) bzw. wei\-ter\-führ\-en\-de Dis\-kus\-sion\-themen plazieren können.
Damit drücken sie ihren Wünsche aus, welche vom Vortragenden live aufgenommen
werden können um den Vortrag zu gestalten.

Tin Can setzt ganz bewusst auf textbasierte Tools um die normale Vortragsebene
nicht zu ersetzen, sondern um eine weitere Ebene zu erweiternt. Damit möchte es
weniger die Teilnahme an einer Diskussion visualiseren und normalisieren, wie es
z.B. System wie Seconde Messenger oder Visiphone machen, sondern das Ergebnis
gebenüber einer klassischen Diskussion verbessern.

\subsection{Studienbeschreibung}
Harry et al. führten Ihre Studie mit zwei Seminaren über 6 Wochen durch. Ein
Gruppe am Morgen mit 8 Studierenden und eine Nachmittagsgruppe mit 11
Studierenden. In die Studie flossen die Beobachtungen der Seminar,
Aufzeichnungen der mit Tin Can erfassten Texte sowie semistrukturierte
Interviews.

Auf eine Video- oder Sprachaufzeichnung wurde verzichtet, damit nicht dieses
Medium als störender Faktor die Studie verfälscht (man wollte ja eben gerade
nicht die Diskussion stören). Die Beobachtung wurde nicht durch den
Semniarleiter aufgezeichnet sondern durch einen neutralen Beobachter da der
Gruppenleiter als Teil der Gruppe gesehen werden muß.

Die Aufzeichnung des Beobachters folgten keinem Schema, es wurde alles
aufgezeichnet, von der Position des Tablets bis hin zu dem Verhalten der
Gruppenteilnehmer untereindander.

Die durchgeführte Analyse der aufgezeichneten Sitzungen ergab folgende Kennzahlen:

\begin{table}[htp]
  \begin{tabular}{ l  l }
    Sitzungsdauer \O & 105 Min.\\
    Ideen  (gesamt) &  839 \\
    
    Themen (erstellt) & 119 \\
    Themen (diskutiert) & 79 \\
    Themen (direkt veröffentlicht) & 72\% \\
    Themen (später veröffentlicht) & 5\% \\ 
    Themen (unveröffentlicht) & 23\% \\
    Themen (Sitzung) \O & 6,5 \\
    Themen (Sitzung) 
  \end{tabular}
  \caption{Kennzahlen der Tin Can Studie}
\end{table}


\subsection{Ergebnis}


\section{Meeting Mediator}


\subsection{Idee und Motivation}


\subsection{Adaption}


\subsection{Studienbeschreibung}


\subsection{Ergebnis}


\section{Vergleich der beiden Ansätze}
Im Vergleich der beiden Studien erkennt man den unterschiedlichen Schwerpunkt.
Tin Can zielt auf eine Maximierung der Themen/Ideen und auf eine Verbesserung
der Moderation und somit auf eine Regulierung der Gruppendynamik.

Meeting Mediator hingegen setzt auf eine Visualisierung des Teilnehmerverhaltens
in Echtzeit allein auf Grundlage der Sprache (Dauer, Lautstärke und
Geschwindigkeit eines Beitrages), mit dem Ziel die Gruppendynamik zu
visualisieren und damit zu einer Verbesserung der Partizipation aller
Gruppenmitglieder beizutragen.

Eine Übersicht der Systemeigenschaften findet man in folgender Tabelle:

\begin{table}[h]
\begin{tabular}{ l | c | c }
   & Tin Can & Meeting Mediator \\
   & \\
  \hline
  Mobile Lösung & ja & ja \\
  Virtuelle Session möglich & ja & - \\
  Soziometrische Sensoren & - & ja \\
  \hline
  Redezeit Gruppenmitglied & - & ja \\
  Redezeit Thema  & ja & - \\
  Redezeit gesamt & ja & - \\
  \hline
  
  Visualierung Dominanz & - & ja \\
  Visualierung Teilnehmer & ja & ja \\
  \hline
  Themenvorrat & ja & - \\
  Ideenvorrat & ja & - \\
  Private Ideen & ja & - \\
  Archivfunktion & ja & - \\
  

\end{tabular}
\caption{Eigenschaften der Systeme im Vergleich}
\end{table}

Bei Tin Can erkennt man in der Studie den Mehrwert für das einzelne
Gruppenmitglied. Dadurch war die Akzeptanz sehr hoch. Durch den Themen- und
Ideenvorrat wurden die Diskussionen lebhafter und vielfältiger. Die
Mög\-lich\-keit nach der Diskussion eine Zusammenfassung zu bekommen, empfanden
die Teilnehmer als sehr positiv, da man nicht das Gefühl hatte auf der
Nebenbühne etwas zu verpassen. Negativ wurde nur die Texteingabe auf dem Tablet
empfunden, da diese doch sehr zeitintensiv war und einige Teilnehmer ihre Ideen
nicht rechtzeitig einbringen konnten.
Tin Can zeigte deutlich, dass es in virtuellen Gruppen weniger genutzt
wird.

Das Konzept des Meeting Mediator wurde bei den Teilnehmern ebenfalls als nicht
störend empfunden, da die soziometrischen Sensoren einfach um den Hals getragen
werden konnten und die Mobiltelefone allgegenwärtig sind und daher nicht als
Werkzeug in Erscheinung traten.

Die einfache Visualisierung der Gesprächsdynamik war für die
Diskussionteilnehmer leicht interpretierbar und konnte von jedem Teilnehmer zur
per\-sön\-lich\-en Ver\-haltens\-ein\-schätz\-ung und möglichen
Verhaltensänderung eingesetzt werden.

Die Studie zeigte, dass mit dem Meeting Mediator virtuelle Gruppen besseren
Zugriff auf soziometrische Aspekte hatten. Damit näherte sich die Qualität von
virtuellen Gruppen der von Realen an.

Da alle soziometrischen Sensoren in Reichweite aller Teilnehmermobiltelefone
sein mussten (auf Grund der Bluetooth-Anbindung), konnte die Studie die
virtuellen Gruppen nur durch einen Vorhang trennen. Dies mindert in unseren
Augen die Aussagekraft der Studie.
Außerdem wurde nur die Sprache zur Analyse herangezogen obwohl der Anspruch an
den sozio-metrischen Sensor in der Studie mindestens noch die Körperbewegungen
beinhaltet hatte.

\section{Zusammenfassung und Ausblick}
Der Blick auf die Tabelle der Eigenschaften zeigt die wenigen Überschneidungen
der beiden Systeme. Das liegt vor allem an den unterschiedlichen Anforderungen
an das jeweilige System.

Unserer Meinung nach wäre als zukünftige Studie ein Hybrid beider Ansätze
wünschenswert, welcher die Eigenschaften verbinden würde. Moderne Tablets
können mittels Webcam und Mikrofon schon einige soziometrische Parameter
ermitteln (z.B. schaut der Benutzer auf das Tablet/in die Webcam? Lautstärke und
Dauer des Sprachbeitrags). Damit würden zusätzliche Kosten für spezielle
Sensorhardware und Entwicklung entfallen. Mit der Nutzung von Tablets bzw.
UMTS/WLAN würde auch die Bluetooth Problematik von Meeting Mediator entfallen.
Ein solches System wäre Out-of-the-Box für verteilte Umgebungen geeignet. Eine
Umsetzung mit HTML5 könnte zu einer einzigen Anwendung für diverse Endgeräte
führen, damit könn\-te die Akzeptanz weiter gesteigert werden. So könn\-te bei
verteilten Ad-hoc Gruppendiskussionen mit jeglichem Endgerät gearbeitet werden.


% Bibliographie entweder direkt hier eingeben...
\begin{thebibliography}{[9]}

\bibitem{HarGorSch2012}
Drew Harry, Eric Gordon, Chris Schmandt.
\newblock Setting the stage for interaction: a tablet application to augment group discussion in a seminar class. 
\newblock {\em In Proceedings of CSCW 2012, ACM Press}, pp. 1071-1080. doi>10.1145/2145204.2145364. 

\bibitem{DiMicco2007}
J. M. DiMicco, K. J. Hollenbach, A. Pandolfo, and W. Bender.
\newblock The Impact of Increased Awareness While Face-to-Face.
\newblock {\em Human-Computer Interaction}, p. 22, 2007.

\bibitem{KimChaHolPent2008}
T. Kim, A. Chang, L. Holland, and A. S. Pentland.
\newblock Meeting Mediator: Enhancing Group Collaboration using Sociometric Feedback. 
\newblock{ \em In Proceedings of CSCW 2008}. ACM Press, 2008.

\bibitem{Goff1959}
E.Goffmann.
\newblock The Presentation of Self in Everday Life.
\newblock {\em Anchor, 1959}

\bibitem{BergKara2009-1}
Tony Bergstrom, Karrie Karahalios. 
\newblock Vote and Be Heard: Adding Back-Channel Cues to Social Mirrors.
\newblock {\em INTERACT 2009} \url {http://social.cs.uiuc.edu/papers/pdfs/bergstrom-interact-2009.pdf} (letzter Zugriff 13.04.2013)

\bibitem{BergKara2009-2}
Karrie Karahalios, Tony Bergstrom. 
\newblock Social Mirrors as Social Signals:Transforming Audio into Graphics.
\newblock {\em IEEE Computer Graphics and Applications 29(5), September/October 2009} \url {http://social.cs.uiuc.edu/papers/pdfs/g5kar.pdf} (letzter Zugriff 13.04.2013)

\bibitem{WP_Likert}
Wikipedia: Die Likert Skala.
\newblock {\em Wikipedia Deutschland} 
\newblock \url {http://de.wikipedia.org/wiki/Likert-Skala} (letzter Zugriff 13.04.2013)

\bibitem{MITbadge}
 MIT Media Laboratory. 
\newblock {\em Sociometric Badges} 
\newblock \url {http://hd.media.mit.edu/badges/} (letzter Zugriff 28.04.2013)

\bibitem{SchuLuk2007}
T. Schümmer, Stephan Lukosch
\newblock{ \em Patterns for Computer-Mediated Interaction}. Jon Wiley \& Sons Ltd., 2007.

\bibitem{Dur1974}
E. Durkheim
\newblock{ \em Sociology and Philosophy}. Free Press, Nov. 1974

\end{thebibliography}


\end{document}
