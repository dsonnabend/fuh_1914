% !TEX encoding = UTF-8 Unicode

\documentclass{seminarvorlage}

\usepackage[utf8]{inputenc}

\begin{document}

% Unbedingt angeben: Titel, Autoren, E-Mail
% Freiwillig: Adresse
\title{Unterstützung von Gruppendiskussionen in Seminaren}
\numberofauthors{2}
\author{
  \alignauthor ds\\
    \email{yyy@yyy.de}
  \alignauthor aw\\
    \email{xxx@xxx.de}
}

\maketitle

\abstract{Diese Arbeit vergleicht zwei Ansätze zur Unterstützung von Gruppendisskussionen.}

\section{Einleitung}
..
\section{Das TinCan System}
\subsection{Idee und Motivation}

\subsection{Umsetzung}
\subsection{(Unsere) Kritik}

\section{Meeting Mediator}
\subsection{Idee und Motivation}
\subsection{Umsetzung}
\subsection{(Unsere) Kritik}
Augenkontakt als Dominanzindikator kann schwer mit IR Sensoren ermittelt werden,
dreht der Sprecher den Kopf, so zeigt seine Brust weiterhin auf die ihm 
gegenübersitzende Person (siehe Figure 3 in)

\section{Second Messenger}
\subsection{Idee und Motivation}
\subsection{Umsetzung}
\subsection{(Unsere) Kritik}

\section{Zusammenfassung}

\section{Perspektive}
Verbesserungsvorschläge? 

Smartphone-Anwendung $\rightarrow$ Hybrid aus TinCan und MM?

% Bibliographie entweder direkt hier eingeben...
\begin{thebibliography}{9}

\bibitem{HarGorSch2012}
Drew Harry, Eric Gordon, Chris Schmandt.
\newblock Setting the stage for interaction: a tablet application to augment group discussion in a seminar class. 
\newblock {\em In Proceedings of CSCW 2012, ACM Press}, pp. 1071-1080. doi>10.1145/2145204.2145364. 

\bibitem{DiMicco2007}
J. M. DiMicco, K. J. Hollenbach, A. Pandolfo, and W. Bender.
\newblock The Impact of Increased Awareness While Face-to-Face.
\newblock {\em Human-Computer Interaction}, p. 22, 2007.

\bibitem{KimChaHolPent2008}
T. Kim, A. Chang, L. Holland, and A. S. Pentland.
\newblock Meeting Mediator: Enhancing Group Collaboration using Sociometric Feedback. 
\newblock{ \em In Proceedings of CSCW 2008}. ACM Press, 2008.

\end{thebibliography}


\end{document}
