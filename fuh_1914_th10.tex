% !TEX encoding = UTF-8 Unicode

\documentclass{seminarvorlage}

\usepackage[ngerman]{babel}
\usepackage[utf8]{inputenc}
\usepackage{graphicx}

% Silbentrennung im Fehlerfall hier angeben.
\hyphenation{Grup-pen-dis-kus-sion}
\hyphenation{kol-lo-bar-i-tive}
\hyphenation{Klas-sen-situ-at-ion}
\hyphenation{Wahr-neh-mungs-eb-ene}
\hyphenation{Man-ag-ment-tech-nik-en}
\hyphenation{Meeting-Mediator}

\begin{document}

% Unbedingt angeben: Titel, Autoren, E-Mail
% Freiwillig: Adresse
\title{Unterstützung von Gruppendiskussionen in Seminaren}
\numberofauthors{2}
\author{
  \alignauthor D. Sonnabend\\
    \email{dv@voelkerts.de}
    \and
  \alignauthor A. Westerhoff\\
    \email{a.westerhoff@online.de}
}

\maketitle

\abstract{ Wir vergleichen anhand der Systeme Tin Can und Meeting Mediator zwei
Ansätze zur Unterstützung und Verbesserung der Diskussion in Gruppen.
Die Gruppengröße divergiert zwischen vier und elf Personen und bildet damit
typische Seminarsituationen ab. Nach einer kurzen Begriffsbestimmung gehen wir
näher auf die Anforderungen ein und beschreiben, warum eine Gruppendiskussion
manchmal nicht so effektiv ist, wie sie es sein könnte und wie dieses durch eine
Steigerung der Partizipation aller Diskussionsteilnehmer verbessert werden kann.
Beide Systeme beschreiben wir detailliert in Aufbau, Umfang und Nutzung um
abschliessend einen direkten Vergleich herstellen zu können.}


\keywords{Computer vermittelte Kommunikation, Unterstützung von
Gruppendiskussion, Backchannel, Tin Can, Meeting Mediator}

\section{Einleitung}
Die Diskussion einer Gruppe, deren Mitglieder sich am selben Ort zur selben Zeit
gegenübersitzen, wird oft als optimale Kommunikationstechnik betrachtet. Harry
et al. \cite{HarGorSch2012} sprechen sogar von einem `Goldstandard`. Viele
Com\-puter-\-ge\-stützte Systeme versuchen, genau diesen
Kommunikationsstandard virtuell abzubilden und eine Dimension zu ersetzen z.B. den unterschiedlichen
Ort oder die fehlende audio-visuelle Kommunikation. Dabei wird oft unterschätzt,
dass dieser 'Goldstandard' ebenfalls weiter verbessert werden kann, denn auch
solche Diskussionen sind häufig nicht so effektiv, wie sie es sein könnten.
Dabei denkt sicherlich jeder an selbst erlebte Besprechungen, die oft auf Grund schlechter
Strukturierung bzw. durch Unterbrechungen wesentlich fragmentiert wurden oder
von einigen Teilnehmern dominiert wurden, so dass wichtige Aspekte in der
Diskussion nicht berücksichtigt wurden.

Einfache Managementtechniken wie Brainstorming, eine Tagesordnung und eine
Rednerliste können da sicherlich bereits helfen.
Die beiden vorgestellten Systeme Tin Can und Meeting Mediator gehen allerdings
noch einen Schritt weiter und versuchen synchron zur Diskussion bzw. dem Vortrag
eine weitere Wahrnehmungsebene zu schaffen.

% Im Falle des Systemes Tin Can
% handelt es sich um Textnotizen, welche das Brainstorming bzw. die Diskussion
% gestalten und führen sollen. Bei dem Meeting Mediator geht es eher um zeitnahe
% Visualisierung der Beteiligung jedes einzelnen Gruppenteilnehmers während der
% Gruppendiskussion. Beide System wollen die Beteiligung steigern und somit
% schneller und effektiver zu Lösungen kommen.

\section{Begriffsbestimmung}
\subsection{Gruppendiskussion}
Unter dem Begriff der Gruppendiskussion verstehen wir in diesem Zusammenhang
eine themenzentrierte Ge\-sprächs\-situa\-tion von mehr als 2 Personen mit dem
Ziel des Brainstorming oder der Problemlösung. Bei einer Gruppendiskussion befinden
sich die Teilnehmer zeitgleich an dem selben Ort oder räumlich verteilt
(virtuelle Gruppe).

\subsection{Awareness}
Awareness ist die Wahrnehmung eines anderen Individuums bzw. dessen Präsenz.
Dazu gehören nicht nur die Sinneswahrnehmungen wie sehen, hören und fühlen,
sondern auch insbesondere die Interpretation dieser Wahrnehmungen im Kontext zu
den eigenen Aktivitäten.
Mit der Herstellung von Awareness in kooperativen Umgebungen mö\-chte man alle
soziologischen Signale transportieren wie z.B. Emotionen und allen
Gruppenmitgliedern den gleichen Kenntnisstand vermitteln.

\subsection{Aufmerksamkeit}
Laut Durkheim \cite{Dur1974} sind wir in jeder Situation immer mehr oder weniger
durch externe Stimuli abgelenkt. Diese Stör\-un\-gen unterbrechen den fiktiven
Begriff der Aufmerksamkeit. Es bedarf also einer Anstrengung anderen Personen
seine Aufmerksamkeit mitzuteilen. Das kann durch verschiedenste soziale Aktionen
präsentiert werden wie z.B. Zuhören oder aber das Schreiben von Notizen.

\subsection{Dominanz}
Tritt ein Individuum aus einer Gruppe durch üb\-er\-durch\-schnitt\-liche
Beteiligung heraus und wirkt damit beherrschend im Diskussionsverlauf, sprechen
wir von Dominanz.
In den beiden vorliegenden Studien wird Dominanz (im Gegensatz zu der
allgemeinen Bedeutung) nicht im negativen Kontext gesehen. Viel mehr beschreibt
eine hohe Dominanz die Beteiligung in der Gruppe.
Die Steigerung der Dominanz bei zurückhaltenderen Gruppenmitgliedern wird oft
als Ziel von Studien genannt.

\subsection{Kanal (Channel)}
Ein Kanal beschreibt in den vorgestellten Studien mehrere Dinge.
\cite{BergKara2009-1}.
Erstens ist ein Kommunikationskanal der Weg einer Botschaft vom Sender zum
Empfänger. Dies ist in der Soziologie die Sprache oder non-verbal die
Kör\-per\-spra\-che.
Im technischen Sinn beschreibt ein Kanal den Über\-mitt\-lungs\-weg von
Nachrichten.
So ist z.B. die Sprache ein Kommunikationskanal, ein Chat ein textueller Kanal
zum Austausch von Informationen. Ein Kanal wird oft von einem einzelnen
Teilnehmer genutzt um viele Teilnehmer zu erreichen (Broadcast).
In Gruppendiskussionen gibt es häufig Front- und Backchannel. Über den
Frontchannel kommuniziert der Vortragende und die direkt an der Diskussion
beteiligten Gruppenmitglieder. Über den Backchannel werden jedoch häufig
zusätzliche Informationen ausgetauscht, wie z.B. Nebendiskussionen oder Notizen.
Häufig stören die Aktivitäten die über den Backchannel laufen, die eigenliche
Frontchannel-Diskussion.

\subsection{Bühne (Stage)}
Anders als Kanäle beschreiben Bühnen (engl. Stage) Orte, an denen
gleichberechtigt eine Performance stattfinden kann \cite{Goff1959}. Dabei ist
weniger der Durchsatz das entscheidende Kriterium. Vielmehr soll ein
zusätzlicher Raum geschaffen werden um Informationen 'abzulegen'. Als Analogie dient den
Autoren Harry et al. ein Restaurant mit Speisesaal (front performance) und Küche
(back performance).
An beiden Orten werden verschiedene Tätigkeiten von den gleichen handelnden
Personen 'performt' die zu einem gemeinsamen Ergebnis führen.

\subsection{Soziale Spiegel (Social Mirrors)}
Soziale Spiegel \cite{BergKara2009-2} sind Visualisierungen von sozialen
Gruppendynamiken in Echtzeit, die allen Gruppenmitgliedern zur Verfügung stehen,
mit dem Ziel Dynamiken zu ver\-änd\-ern. Visualisiert wird das eigene Verhalten
sowie das Verhalten aller anderen Gruppenmitglieder aus der Sicht einer dritten
Person. Dabei wird die Relation des Individuums zur Gruppe verdeutlicht. Dies
soll jedem Individuum Anreize bieten, sein Verhalten an den Gruppendurchschnitt
anzupassen. Verarbeitet werden insbesondere auditive und visuelle, aber
auch emotionale Signale.

\section{Problemanalyse \& Anforderungen}
In Arbeitsgruppen wie zum Beispiel Seminaren, münd\-lich\-en Diskussionen oder
Projektbesprechungen gibt es Barrieren, die die Partizipation aller
Mitglieder verhindern und damit die Effektivität der Gruppe einschränken.

Dadurch kommt die Gruppe entweder gar nicht bzw. nicht effektiv zu einem
best\-mög\-lich\-en Ergebnis. Durch Dominanzen in der Gruppe kommt es nicht zu
einer gleichberechtigten Beteiligung allr Gruppenmitglieder an der Diskussion.
Dies mindert die Themenvielfalt und den Ideenaustausch. Introvertierte
Per\-sön\-lich\-keit\-en werden nicht maximal gefördert.

Gefordert wird, Dominanzen zu erkennen, introvertierte Per\-sön\-lich\-keit\-en
zu couragieren und die Themen-/\-Ideen\-vielfalt zu maximieren.

\section{Tin Can}


\subsection{Idee und Motivation}

\subsection{Adaption}

\subsection{Studienbeschreibung}


\subsection{Ergebnis}


\section{Meeting Mediator}
%Autor: Antje Westerhoff

\subsection{Idee und Motivation}
Meeting Mediator(MM) \cite{KimChaHolPent2008} wurde als System zur
Un\-ter\-stütz\-ung von Gruppendiskussionen in räumlich gleichen und verteilten
Arbeitsumgebungen entwickelt. Das System liefert ein real-time Feedback auf der
Basis von Daten, die mittels soziometrischer Aufzeichnungen ermittelt werden.
Dieses Feedback soll Gruppen unterstützen, Diskussionen effektiver zu führen,
indem alle Mitglieder gleichermaßen an der Diskussion teilnehmen.

Diskussionsgruppen setzen sich meist aus unterschiedlich dominanten Mitgliedern
zusammen. Die Dominanz einzelner Gruppenmitglieder führt zwar einerseits die
Diskussion voran, andererseits werden weniger dominante Diskussionsteilnehmer 
entmutigt, oder deren Beiträge un\-ter\-drückt.

Insbesondere bei Diskussionen mit verteilten Gruppenmitgliedern fehlt der
Gruppe teilweise die Möglichkeit auf die zusätzlich zum gesprochenen Wort
vorhandenen verschiedenen soziologischen Signale, z.B. Enthusiasmus oder
Kör\-per\-hal\-tung, der anderen Gruppenmitglieder zu reagieren.
Die von MM verwendeten soziometrischen Sensoren (Sociometric badges)
\cite{MITbadge} sammeln und analysieren verbale und nonverbale
Verhaltensmerkmale.

Mittels MM werden diese Daten ausgewertet und den jeweiligen Gruppenmitgliedern
auf deren Mobiltelefon visualisiert. Dieses Vorgehen soll dem einzelnen
Diskussionsteilnehmer eine wenig ablenkende Information bieten, das eigene
Verhalten im Kontext zum Gruppenverhalten einzuordnen und daraus möglicherweise
Ver\-hal\-tens\-än\-de\-run\-gen abzuleiten, um den Gruppenerfolg zu
vergrößeren.

\subsection{Adaption}

MM ist eine J2ME-Applikation für Smartphones. J2ME \cite{J2ME} ist die Micro
Edition von Java (Java ME), die speziell die Erstellung von Java-Anwendungen
ermöglicht, die auf kleinen Geräten mit begrenzter Speicher-, Anzeige- und
Leistungskapazität ausgeführt werden. Die zur Software-Ent\-wick\-lung zur
Verfügung stehende API (Application Programming Interface) und ein umfangreiches
SDK (Software Development Kit) ermöglichen, durch die Nutzung der Java VM, eine
hardware-un\-ab\-hän\-gi\-ge Programmentwicklung.

Zu\-sätz\-lich werden Sociometric Badges \cite{MITbadge} als dezentrale
Einheiten zur Aufzeichnung und Auswertung soziologischer Daten eingesetzt.
Die Geräte erkennen Audio-Signale und messen mittels Infrarot-Sen\-so\-ren
Bewegungen und ü\-ber\-tra\-gen diese in Echtzeit.
Die Studie \cite{KimChaHolPent2008} wertet zur Erkennung von dominanten
Teilnehmern die Sprache und die Kör\-per\-be\-we\-gun\-gen aus. Die Län\-ge und Häu\-fig\-keit von
Ge\-sprächs\-bei\-trä\-gen und die Bewegung und Hinwendung zu anderen
Gruppenmitgliedern charakterisieren die Dominanz von Diskussionsteilnehmern.
 
Um Teilnehmern an einer Gruppendiskussion zusätzliche Informationen über
soziologische Signale zur Verfügung stellen zu können, trägt jeder Teilnehmer
einen mobilen Sensor (Sociometric badges) um den Hals (siehe Abbildung
\ref{badge}).
 Mittels Bluetooth werden die Messdaten an das Mobiltelefon des
Teilnehmers übermittelt. Aufgrund der Reichweite von Bluetooth wurde die
räumlich verteilte Gruppe nur durch einen Sichtschutz zwischen den
Gruppenteilnehmern simuliert.

\begin{figure}[htp]
\centering
\includegraphics[width=8cm]{sociometricbadge.jpg}
\caption{Soziometrischer Sensor (Sociometric Badge) \protect\cite{MITbadge}}
\label{badge}
\end{figure}

Die so übermittelten Daten werden dann durch die Meeting-Mediator-Applikation
akkumuliert und visualisiert.
Die graphische Anzeige auf den Mobiltelefonen der Teilnehmer wird in 5 Sekunden
Intervallen aktualisiert und kumuliert die Daten über den gesamten Zeitraum der
Diskussion.
Die Anzeige auf den Mobiltelefonen der Diskussionsteilnehmer wurde gewählt, um
nicht einen weiteren, mög\-li\-cher\-wei\-se ablenkenden, Kommunikationskanal
ein\-zu\-füh\-ren.

Die auf dem Mobiltelefon angezeigte Grafik (siehe Abbildung \ref{display}),
stellt die einzelnen Mitglieder als Rechtecke in unterschiedlichen Farben,
korrespondierend zu den Farben ihrer Sociometric badges, dar. Die Balance der
Diskussion wird als zentraler Kreis dargestellt, der wie ein Ball an elastischen
Seilen zu den Rechtecken der Diskussionsteilnehmer gezogen wird, abhängig von
deren jeweiligem Anteil an der Diskussion. Die Dicke der Linie, die den
zentralen Kreis mit den Rechtecken verbindet, visualisiert die Sprechzeit des
jeweiligen Teilnehmers. Die Farbe des zentralen Kreises zeigt mit grün eine hohe
Interaktivität der Diskussion an und möchte mit weiß zu mehr Interaktion
ermuntern.

\begin{figure}[htp] 
\centering 
\includegraphics[width=8cm]{mm.jpg}
\caption{Benutzeroberfläche von Meeting Mediator
\protect\cite{KimChaHolPent2008}}
\label{display}
\end{figure}


\subsection{Evaluation}
MM möchte die Effizienz von Gruppendiskussionen er\-höh\-en, indem die
Gruppendynamik in Echtzeit graphisch dargestellt wird und, damit die Änderung
von individuellen und Gruppenverhaltensweisen fördern.
 
Die MM Entwickler Kim et. al. führten eine Studie \cite{KimChaHolPent2008} an 36
Gruppen mit je 4 Gruppenmitgliedern zur Bewertung von MM durch. Wegen eines
technischen Defekts eines Sociometric Badges bei der Audio-Analyse wurden
jedoch nur jeweils 3 Gruppenmitglieder zur tatsächlichen Auswertung
herangezogen.
Eine Hälfte, also 18 Teams, führten die Aufgaben mit Unterstützung durch ein
MM-Feedback durch, während die anderen 18 Gruppen als Kontrollgruppen ohne MM
arbeiteten.

Die Aufgaben aller Teams mussten jeweils in 2 unterschiedlichen Situationen
bearbeitet werden.
Die erste Situation bildete den Fall ab, dass alle Mitglieder des Teams sich zur
gleichen Zeit am selben Ort befinden. Die zweite Situation simulierte den Fall,
dass sich je 2 Mitglieder zur gleichen Zeit an unterschiedlichen Orten,
befinden.
Die räumliche Trennung wurde für die Studie nur durch eine Sichttrennung mittels
eines Vorhangs simuliert.
Unabhängig von der jeweiligen Situation, sollten 2 unterschiedliche
Aufgabenblöcke bearbeitet werden. Im ersten Block ging es um die Ideensammlung
und im zweiten Block um die Pro\-blem\-lö\-sung, von jeweils gleichen Problemen.

Zu jeder Situation wurde im Anschluss von jedem Gruppenmitglied ein persönlicher
Fragebogen mit einer Fünf- Punkte Likert-Skala \cite{WP_Likert} zu den Fragebereichen eigener
Charakter, Gruppendynamik und Leistung der Gruppenmitglieder und, falls
verfügbar, zum Nutzen von MM bearbeitet.
 
Die Studie zeigte, dass:

\begin{itemize}
  \item MM dazu beiträgt, dass weniger überlappende Gespräche geführt werden, d.h. die
Gruppe teilt sich nicht in Untergruppen, sondern diskutiert gemeinsam.
  \item MM zu kürzeren Sprechsequenzen führt. Es findet insbesondere in verteilten
Umgebungen eine höhere Interaktion statt.
  \item die Nutzung des allgegenwärtigen Mobiltelefons als Anzeigemedium die
Studienteilnehmer weniger ablenkt und stört als eine für alle sichtbare
zusätzliche Anzeige.

\end{itemize}

Die von den MM Entwicklern aufgestellten Hypothesen zur Identifikation und
Beeinflussung von dominanten und zurückhaltenderen Gruppenmitgliedern zur
Förderung des kooperativen Arbeitens, sind nicht so eindeutig bestätigt worden,
wie die zuvor beschriebenen Effekte.

Durch die Messung der Sprechzeit identifizierte MM dominante Gruppenmitglieder
genauso, wie es auch intuitiv erfolgen würde. Es fand sich aber kein
signifikanter Unterschied zwischen dominanten und weniger dominanten
Gruppenmitgliedern bei der Messung von Bewegungen.
Der Einfluss, den MM auf das Verhalten von dominanten und weniger dominanten
Diskussionsmitgliedern hat, ist sehr unterschiedlich.
Bei der Beobachtung ohne die Beeinflussung von MM zeigte sich, dass weniger
dominante Personen in Gruppen mit einer dominanten Person insbesondere in
verteilten Arbeitsgruppen häufiger das Wort ergreifen, als in Gruppen mit
ausschließlich wenig-do\-min\-ant\-en Personen.

Die Hypothese, dass MM die Gruppendiskussionen in verteilten Umgebungen
wesentlich mehr beeinflusst, als in räumlich nicht getrennten Umgebungen, konnte
nicht allgemein bestätigt werden. Hier wurde eher ein Zusammenhang mit dem
Vorhandensein von dominanten Personen in der Gruppendiskussion hergestellt.
Insbesondere in verteilten Umgebungen fällt es weniger-dominanten Personen mit
MM leichter, sich in die Diskussion einzubringen, da MM den dominanten
Gruppenmitgliedern die fehlende Balance in der Diskussion optisch vermitteln
kann. 
Der Einsatz von MM zeigte, dass man durch die Visualierung des Gruppenverhaltens
die Unterschiede im Verhalten von dominanten und weniger dominanten
Diskussionsteilnehmern verringern kann.

\subsection{Ergebnis}
MM ist ein mobiles System, das ein real-time feedback in Gruppendiskussionen
liefert, mit dem Ergebnis, dass die Zusammenarbeit verbessert wird.
Dieser Effekt macht sich insbesondere in verteilten Gruppendiskussionen mit
dominanten Gruppenmitgliedern bemerkbar. Verteilte Diskussionen werden damit
lokalen Diskussionen angenähert.




\section{Vergleich der beiden Ansätze}
Im Vergleich der beiden Studien erkennt man den unterschiedlichen Schwerpunkt.
Tin Can zielt auf eine Maximierung der Themen/Ideen und auf eine Verbesserung
der Moderation und somit auf eine Regulierung der Gruppendynamik.

Meeting Mediator hingegen setzt auf eine Visualisierung des Teilnehmerverhaltens
in Echtzeit allein auf Grundlage der Sprache (Dauer, Lautstärke und
Geschwindigkeit eines Beitrages), mit dem Ziel die Gruppendynamik zu
visualisieren und damit zu einer Verbesserung der Partizipation aller
Gruppenmitglieder beizutragen.

Eine Übersicht der Systemeigenschaften findet man in folgender Tabelle:

\begin{table}[h]
\begin{tabular}{ l | c | c }
   & Tin Can & Meeting Mediator \\
   & \\
  \hline
  Mobile Lösung & ja & ja \\
  Virtuelle Session möglich & ja & - \\
  Soziometrische Sensoren & - & ja \\
  \hline
  Redezeit Gruppenmitglied & - & ja \\
  Redezeit Thema  & ja & - \\
  Redezeit gesamt & ja & - \\
  \hline
  
  Visualierung Dominanz & - & ja \\
  Visualierung Teilnehmer & ja & ja \\
  \hline
  Themenvorrat & ja & - \\
  Ideenvorrat & ja & - \\
  Private Ideen & ja & - \\
  Archivfunktion & ja & - \\
  

\end{tabular}
\caption{Eigenschaften der Systeme im Vergleich}
\end{table}

Bei Tin Can erkennt man in der Studie den Mehrwert für das einzelne
Gruppenmitglied. Dadurch war die Akzeptanz sehr hoch. Durch den Themen- und
Ideenvorrat wurden die Diskussionen lebhafter und vielfältiger. Die
Mög\-lich\-keit nach der Diskussion eine Zusammenfassung zu bekommen, empfanden
die Teilnehmer als sehr positiv, da man nicht das Gefühl hatte auf der
Nebenbühne etwas zu verpassen. Negativ wurde nur die Texteingabe auf dem Tablet
empfunden, da diese doch sehr zeitintensiv war und einige Teilnehmer ihre Ideen
nicht rechtzeitig einbringen konnten.
Tin Can zeigte deutlich, dass es in virtuellen Gruppen weniger genutzt
wird.

Das Konzept des Meeting Mediator wurde bei den Teilnehmern ebenfalls als nicht
störend empfunden, da die soziometrischen Sensoren einfach um den Hals getragen
werden konnten und die Mobiltelefone allgegenwärtig sind und daher nicht als
Werkzeug in Erscheinung traten.

Die einfache Visualisierung der Gesprächsdynamik war für die
Diskussionteilnehmer leicht interpretierbar und konnte von jedem Teilnehmer zur
per\-sön\-lich\-en Ver\-haltens\-ein\-schätz\-ung und möglichen
Verhaltensänderung eingesetzt werden.

Die Studie zeigte, dass mit dem Meeting Mediator virtuelle Gruppen besseren
Zugriff auf soziometrische Aspekte hatten. Damit näherte sich die Qualität von
virtuellen Gruppen der von Realen an.

Da alle soziometrischen Sensoren in Reichweite aller Teilnehmermobiltelefone
sein mussten (auf Grund der Bluetooth-Anbindung), konnte die Studie die
virtuellen Gruppen nur durch einen Vorhang trennen. Dies mindert in unseren
Augen die Aussagekraft der Studie.
Außerdem wurde nur die Sprache zur Analyse herangezogen obwohl der Anspruch an
den sozio-metrischen Sensor in der Studie mindestens noch die Körperbewegungen
beinhaltet hatte.

\section{Zusammenfassung und Ausblick}
Der Blick auf die Tabelle der Eigenschaften zeigt die wenigen Überschneidungen
der beiden Systeme. Das liegt vor allem an den unterschiedlichen Anforderungen
an das jeweilige System.

Unserer Meinung nach wäre als zukünftige Studie ein Hybrid beider Ansätze
wünschenswert, welcher die Eigenschaften verbinden würde. 

Moderne Tablets können mittels Webcam und Mikrofon schon einige soziometrische
Parameter ermitteln (z.B. schaut der Benutzer auf das Tablet/in die Webcam?
Lautstärke und Dauer des Sprachbeitrags). Damit würden zusätzliche Kosten für
spezielle Sensorhardware und Entwicklung entfallen. Mit der Nutzung von Tablets
bzw.
UMTS/WLAN würde auch die Bluetooth Problematik von Meeting Mediator entfallen.
Ein solches System wäre Out-of-the-Box für verteilte Umgebungen geeignet.

Eine Umsetzung mit HTML5 könnte zu einer einzigen Anwendung für diverse
Endgeräte führen, damit könn\-te die Akzeptanz weiter gesteigert werden. So
könn\-te bei verteilten Ad-hoc Gruppendiskussionen mit jeglichem Endgerät
gearbeitet werden.


% Bibliographie entweder direkt hier eingeben...
\begin{thebibliography}{[9]}

\bibitem{HarGorSch2012}
Drew Harry, Eric Gordon, Chris Schmandt.
\newblock Setting the stage for interaction: a tablet application to augment group discussion in a seminar class. 
\newblock {\em In Proceedings of CSCW 2012, ACM Press}, pp. 1071-1080. doi>10.1145/2145204.2145364. 

\bibitem{DiMicco2007}
J. M. DiMicco, K. J. Hollenbach, A. Pandolfo, and W. Bender.
\newblock The Impact of Increased Awareness While Face-to-Face.
\newblock {\em Human-Computer Interaction}, p. 22, 2007.

\bibitem{KimChaHolPent2008}
T. Kim, A. Chang, L. Holland, and A. S. Pentland.
\newblock Meeting Mediator: Enhancing Group Collaboration using Sociometric Feedback. 
\newblock{ \em In Proceedings of CSCW 2008}. ACM Press, 2008.

\bibitem{Goff1959}
E.Goffmann.
\newblock The Presentation of Self in Everday Life.
\newblock {\em Anchor, 1959}

\bibitem{BergKara2009-1}
Tony Bergstrom, Karrie Karahalios. 
\newblock Vote and Be Heard: Adding Back-Channel Cues to Social Mirrors.
\newblock {\em INTERACT 2009} \url {http://social.cs.uiuc.edu/papers/pdfs/bergstrom-interact-2009.pdf} (letzter Zugriff 13.04.2013)

\bibitem{BergKara2009-2}
Karrie Karahalios, Tony Bergstrom. 
\newblock Social Mirrors as Social Signals:Transforming Audio into Graphics.
\newblock {\em IEEE Computer Graphics and Applications 29(5), September/October 2009} \url {http://social.cs.uiuc.edu/papers/pdfs/g5kar.pdf} (letzter Zugriff 13.04.2013)

\bibitem{WP_Likert}
Wikipedia: Die Likert Skala.
\newblock {\em Wikipedia Deutschland} 
\newblock \url {http://de.wikipedia.org/wiki/Likert-Skala} (letzter Zugriff 13.04.2013)

\bibitem{MITbadge}
 MIT Media Laboratory. 
\newblock {\em Sociometric Badges} 
\newblock \url {http://hd.media.mit.edu/badges/} (letzter Zugriff 28.04.2013)

\bibitem{SchuLuk2007}
T. Schümmer, Stephan Lukosch
\newblock{ \em Patterns for Computer-Mediated Interaction}. Jon Wiley \& Sons Ltd., 2007.

\bibitem{Dur1974}
E. Durkheim
\newblock{ \em Sociology and Philosophy}. Free Press, Nov. 1974

\end{thebibliography}


\end{document}
