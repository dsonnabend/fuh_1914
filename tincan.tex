\section{Tin Can}

\subsection{Idee und Motivation}
Das Tin Can System \cite{HarGorSch2012} wurde 2012 am MIT Media Lab entwickelt.
Dabei handelt es sich um ein Tablet basiertes kollobaritive System, welches in
einer typischen Klassensituation es ermögicht, Ideen und wei\-ter\-führ\-en\-den
Diskussionen zu sammeln.

Motiviert durch zahlreiche Studien, welche belegen, dass eher schüchterne
Persönlichkeiten von zusätzlichen kollobrorativen Werkzeugen profitieren, möchte
Tin Can die Teilnahme am Unterricht bzw. Vortrag fördern.

Die Ideensammlung bzw. Themensammlung für weitere Diskussionen wird sowohl live
genutzt, damit der Vortragende mehr auf das Auditorium eingehen kann und es so
zu einer verbesserten Moderation kommt, als auch um nach\-träglich eine Analyse
des Vortrages zu erhalten. So erhält zum Beispiel jeder Teilnehmer eine E-Mail
nach Ende des Vortrages mit einer Ideen- und Themenliste.

Dadurch dass der Dozent den Backchannel moderierend beobachtet, kann er auf
Themen dediziert eingehen oder 'schüchterne' Teilnehmer ermutigen Ihre privaten
Notizen vorzurtragen oder zu veröffentlichen.

Im Unterschied zu anderen Arbeiten (Referenzen!), ist bei TinCan der Backchannel
dazu gedacht, den Frontchanel, also den Vortragenden oder Moderierenden zu
beeinflussen und so der Diskussion eine neue Richtung zu geben.

Die meisten Arbeiten versuchen den Übergang zwischen Front- und Backchannel zu
optimieren, die Motivation bei Tin Can hingegen liegt hingegeen auf der
Einführung eines weiteren Komzeptes, die s.g. Stages.

Dabei sind Stages am ehesten als eine Art Bühne zu verstehen. Neben der
Hauptbühne (dem Dozenten) gibt es eine Nebenbühne, das Tin Can System.

Tin Can ist kein Broadcast System in dem ein Teilnehmer leicht viele Teilnehmer
erreichen kann (und evenentuell mit der dadurch entstehenden Dominanz andere
Teilnehmer unterdrückt), sondern eher eine Nebenraum, in dem viele
Persönlichkeiten mit Ihrer 'Zielgruppe' kommunizieren können. Das kann der
Frontchannel sein, aber auch eine private Idee oder aber ein Themenvorschlag.

Im Gegensatz zu der Channelmethapher sollen die Teilnehmer motiviert werden
viele kleine Beiträge beizusteuern und nicht große Broadcast.

Die Tin Can Studie geht auf Grundlage moderner Sozialpsychologie davon aus, dass
introvertierte Persönlichkeiten eher schriftlich an einer Diksussion teilnehmen,
da sie so vorher ihren Beitrag prüfen, korrigieren oder ganz verwerfen können.

Daraus schliessen sie, dass der goldene Standard der normalen Diksussion in
kleinen Seminargruppen verbessert werden kann, in dem man durch einen
schriftliche Beteiligungsmöglichkeit auch die introvertieren Gruppenmitglieder
zu Wort kommen lässt. Somit verbessert sich das Outcome der Diskussion.
 
\subsection{Architektur \& Technische Umsetzung}
Tin Can wurde als eine App für Tablets umgesetzt, da Tablets unauffällige
Begleiter im Alltag sind und weniger störend wirken wie ein aufgeklapptes
Laptop.

Durch die geringe Formgröße kann jedes Gruppenmitglied selbst
entscheiden wie es das Gerät halten möchte. Damit reduziert sich die
Wahrnehmung des Gerätes als eigenständigen Gegenstand. Es ist mehr wie ein
Schreibblock.

Tin Can wurde als Client-Server-Anwendung umgesetzt, die Clients sind reine
Tablets, der Server führt ein Archive und bietet nachgelagerte Dienste an, wie
z.B. den Mailversand mit einer Zusammenfassung der Diskussion nachdem diese
beendet wurde.

Bei Tin Can gibt es kein Rechtesystem, alle Benutzer sind gleichberechtigt.

Das System zeigt den aktuellen Stand der Diskussion. Der Bildschirm zeigt 3
Hauptbereiche: eine Themenliste, eine analoge Uhrzeit und eine Ideenliste.

Die Themenliste sammelt sowohl alte, aktuellen und vorgeschlagene Themen, jeder
Benutzer kann einfach mit einem Button 'Theme hinzufügen' ein neues Thema
erstellen. Jedes Thema wird mit einem Kurztext beschrieben und mit einer Farbe
von anderen Themen abgegrenzt. Anhand einer kleinen Uhr kann man die Dauer der
diskutierten Themen sehen, bzw. die aktuelle Dauer bei dem aktuellen Thema.

Indem man ein Thema antippt, kann man in einem Untermenü den Status des Themas
verändern.

Die analoge Uhr zeigt erstmal die aktuelle Uhrzeit, sowie farblich markiert die
Dauer der einzelnen Themen.
Wenn eine Stunde verstrichen ist, werden die Segemente welche die Dauer
visualisieren aus dem Zentrum der Uhr an ihre Ecken ausgelagert und die aktuelle
Stunde wird im Zentrum dargestellt. So erreicht Tin Can eine maximale
Sitzungsdauer von 4 Std. ohne dass die Visualisierung unübersichtlich wird.

Die Liste der Ideen befindet sich auf der rechten Seite des Bildschirmes, die
Ideen sind chronologisch sortiert. Ideen sind einfach Textbeiträge wie Fragen,
Aussagen oder Twitter-ähnliche Antworten. Man kann mit zwei Buttons eine Idee
hinzufügen bzw. zur Gruppe hinzufügen. Erster Button speichert die Idee in einem
persönlichen Bereich, die anderen pupliziert sofort die Idee im Zeitraster der
Gruppenideen und zu dem persönlichen Bereich. Ideen welche im Gruppenraster
dargestellt werden, weisen in Klammern den Autor des Beitrages aus. Außerdem
gibt es ein Bewertungssystem, mit welchem andere Teilnehmer Ihre Zustimmung
signalisieren können. Wie man auch in anderen Studien gesehen hat, so nutzt auch
Tin Can nur ein positives Feedback in Facebook-Form (Likes).

Benutzer werden am Rand des Bildschirm nach einem Zufallsprinzip als Rechtecke
mit Ihrem Namen dargestellt. Mit einem Tipp auf diese Rechteck kann man alle
Ideen (auch die persönlichen) des Benutzer sehen. Jeder (nicht nur der
Ersteller) kann eine Idee eines anderen Benutzer aus dessen persönlichen
Bereiches in die Gruppenliste der Ideen ziehen. Diese Idee wird dann
entsprechend gekennzeichnet ('Eine Idee von Daniel veröffentlich von Antje).

Damit sind persönliche Ideen per Definition nichts privates. Sie bieten dem
Benutzer eher eine Möglichkeit, seine Zurückhaltung bei einer Idee auszudrücken.

\subsection{Adaption}
Tin Can führt dazu so genannte `Stages` ein. Diese sind am ehesten mit dem Bild
einer Theaterbühne zu vergleichen. Der Sprecher steht dabei vor der Gruppe auf
der so genannten Haupt-Bühne(`main spoken stage). Die Zuhörer haben in der
normalen Vortragssituation keine Möglichkeit im Vortag Präsenz zu zeigen ohne
unterbrechend einzugreifen. Dazu bietet TinCan eine Nebenbühne bzw. einen
zweiten Kanal (`backchannel`) in welchem die Zuhörer Ihre Ideen (privat wie
öffentlich) bzw. wei\-ter\-führ\-en\-de Dis\-kus\-sion\-themen plazieren können.
Damit drücken sie ihren Wünsche aus, welche vom Vortragenden live aufgenommen
werden können um den Vortrag zu gestalten.

Tin Can setzt ganz bewusst auf textbasierte Tools um die normale Vortragsebene
nicht zu ersetzen, sondern um eine weitere Ebene zu erweiternt. Damit möchte es
weniger die Teilnahme an einer Diskussion visualiseren und normalisieren, wie es
z.B. System wie Seconde Messenger oder Visiphone machen, sondern das Ergebnis
gebenüber einer klassischen Diskussion verbessern.

\subsection{Evaluation}
Harry et al. führten Ihre Studie mit zwei Seminaren über 6 Wochen durch. Ein
Gruppe am Morgen mit 8 Studierenden und eine Nachmittagsgruppe mit 11
Studierenden. In die Studie flossen die Beobachtungen der Seminar,
Aufzeichnungen der mit Tin Can erfassten Texte sowie semistrukturierte
Interviews.

Auf eine Video- oder Sprachaufzeichnung wurde verzichtet, damit nicht dieses
Medium als störender Faktor die Studie verfälscht (man wollte ja eben gerade
nicht die Diskussion stören). Die Beobachtung wurde nicht durch den
Semniarleiter aufgezeichnet sondern durch einen neutralen Beobachter da der
Gruppenleiter als Teil der Gruppe gesehen werden muß.

Die Aufzeichnung des Beobachters folgten keinem Schema, es wurde alles
aufgezeichnet, von der Position des Tablets bis hin zu dem Verhalten der
Gruppenteilnehmer untereindander.


\subsection{Ergebnis}
Die durchgeführte Analyse der aufgezeichneten Sitzungen ergab folgende Kennzahlen:

\begin{table}[htp]
  \begin{tabular}{ l  l }
    Sitzungsdauer \O & 105 Min.\\
    \\
    Ideen  (gesamt) &  839 \\
    Ideen  (veröffentlicht durch Prof.) & 57\% \\
    \\
    Themen (erstellt) & 119 \\
    Themen (diskutiert) & 79 \\
    Themen (direkt veröffentlicht) & 72\% \\
    Themen (später veröffentlicht) & 5\% \\ 
    Themen (unveröffentlicht) & 23\% \\
    \\
    Themen (Sitzung) \O & 6,5 \\
    Themen (Sitzung) $\pm$ & 2,3 \\
    Themendauer \O & 851 Sek. \\
    Themendauer $\pm$ & 673 Sek. \\
  \end{tabular}
  \caption{Kennzahlen der Tin Can Studie}
\end{table}

Auffallend waren hier zwei Kennzahlen: Die hohe Abweichung der Themendauer von
673  sec wurde von den Autoren dadurch erklärt, dass es durch viele kurze Themen
hier zu einer Verzerrung der Verteilung kommt.

Die zweite Abnomalie war die ungleichmäßige Verteilung von Themen gesehen über
die Zeit. Die Studie kommt zu dem Ergebnis, dass dies eine vorhersehbare 'bursty
structure' Verteilung im klassischen Sinne des Begriffs von Kleinberg ist.

In den durchgeführten Interviews wurde klar, dass Tin Can eine zusätzliche
Möglichkeit der Diskussion bietet, genau so, wie es die Autoren mit der
Second-Stage im Sinn hatten. Tin Can bietet die Möglichkeit einer Nebenbühne,
nicht aber den Zwang diese zu benutzen. Das Hauptproblem für die
Studienteilnehmer war die Balance zwischen der Main- und Second-Stage zu finden
ohne sofort unaufmerksam zu wirken.

Ebenso wurde als positiv empfunden, dass man die Diskussion auf der Second-Stage
nicht unbedingt verfolgen musste, da man ja nach Ende der Diskussion eine
Zusammenfassung gemailt bekommt.

Der Dozent nutzte Tin Can oft um das Interesse an seinem Vortrag live zu messen
und Moderation gezielter zu führen.

Negativ galt bei den Teilnehmern die Texteingabe auf dem Tablet, welche oft als
nicht schnell genug empfunden wurde um mit der Diskussion mithalten zu können.

Die Studie zeigt, dass das duale Stagekonzept schüchterne eher als 'Denker'
beschriebene Charaktere die Chance gibt, Ihren Beitrag auf der Side-Stage zu
posten und so an der Diskussion teilzunehmen. Dabei ist nicht immer mangelndes
Selbstvertrauen in den Interviews genannt worden, sondern oft auch der Wunsch
nicht sötrend in den Hauptvortrag eingreifen zu wollen. So führte eine lebhafte
Diskussion auf der virtuellen Bühne zu keiner physikalischen Störung.

Die Möglichkeit Ideen zu teilen, eigene Ideenlisten zu führen und Ideen anderer
aufzugreifen wurde besonders von den Dozenten geschätzt um die Beteiligung durch
direkte Ansprache der Teilnehmer zu erhöhen. Dieser wurde durch die direkte
Integration ebenso wie die 'Likes' von anderen Teilnehmern motiviert, Ihre
privaten Ideen zu erläutern.

Die Studie belegt, dass beide Ziele des Projekts Tin Can erfolgreich umgesetzt
werden konnten, da sowohl Ideen- und Themenvielfalt und Diskussionsteilnahme
verbessert werden konnten.

Allerdings kommt die Studie auch zu dem Schluss, dass es eine kritische Masse im
Sinne der Gruppengröße gibt. In zu kleinen Gruppen kann Tin Can nicht richtig
funktionieren, da die Nebenbühne einen Teil der gesamten Aufmerksamkeit fordert.
In größeren Gruppen kann dieser Teil der Aufmerksamkeit dadurch gewonnen werden,
dass nicht immer jeder gleichzeitig sprechen kann und die Wechselzeiten höher
sind als bei kleineren Gruppen.





