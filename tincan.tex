\section{Tin Can}
Autor: Daniel Sonnabend
max. 1 Seite
\subsection{Idee und Motivation}
Das Tin Can System\cite{HarGorSch2012} wurde 2012 am MIT Media Lab entwickelt.
Dabei handelt es sich um ein Tablet basiertes kollobartive System, welches in
einer typischen Klassensituation es ermögicht, Ideen und weiterführenden
Diskussionen zu sammeln.

Motiviert durch zahlreiche Studien, welche belegen, dass eher schüchterne
Persönlichkeiten von zusätzlichen kollobrorativen Werkzeugen profitieren, möchte
Tin Can die Teilnahme am Unterricht bzw. Vortrag fördern.

Die Ideensammlung bzw. Themensammlung für weitere Diskussionen wird sowohl live
genutzt, damit der Vortragende mehr auf das Auditorium eingehen kann und es so
zu einer verbesserten Moderation kommt, als auch um nach\-träglich eine Analyse
des Vortrages zu erhalten. So erhält zum Beispiel jeder Teilnehmer eine E-Mail
nach Ende des Vortrages mit einer Ideen- und Themenliste.

\subsection{Adaption}
Tin Can führt dazu so genannte `Stages` ein. Diese sind am ehesten mit dem Bild
einer Theaterbühne zu vergleichen. Der Sprecher steht dabei vor der Gruppe auf
der so genannten Haupt-Bühne(`main spoken stage). Die Zuhörer haben in der
normalen Vortragssituation keine Möglichkeit im Vortag präsenz zu zeigen ohne
unterbrechend einzugreifen. Dazu bietet TinCan eine Nebenbühne bzw. einen
zweiten Kanal (`backchannel`) in welchem die Zuhörer Ihre Ideen (privat wie
öffentlich) bzw. wei\-ter\-führ\-en\-de Dis\-kus\-sion\-themen plazieren können.
Damit drücken sie ihren Wünsche aus, welche vom Vortragenden live aufgenommen
werden können um den Vortrag zu gestalten.

Tin Can setzt ganz bewusst auf textbasierte Tools um die normale Vortragsebene
nicht zu ersetzen, sondern um eine weitere Ebene zu erweiternt. Damit möchte es
weniger die Teilnahme an einer Diskussion visualiseren und normalisieren, wie es
z.B. System wie Seconde Messenger oder Visiphone machen, sondern das Ergebnis
gebenüber einer klassischen Diskussion verbessern.

\subsection{Studienbeschreibung}


\subsection{Ergebnis}

